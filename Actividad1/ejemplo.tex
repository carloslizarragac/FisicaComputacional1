% Ejemplo de un documento en LaTeX. 
% LaTeX toma como comentario cualquier texto a la derecha del símbolo % 
% La notación en LaTeX es:  \nombre_de_la_función[argumentos]{opciones}

% \documentclass{} Es el primer comando obligatorio en cualquier documento.  
% Ademas de article, puede ser report o book. 
\documentclass{article} 
                                     
 
% Bibliotecas adicionales a incluir. 
% \usepackage es la forma de añadir bibliotecas adicionales a los documentos LaTeX.
% En ese caso para añadir caracteres en Español (á, é, í, ó, ú, ñ, ü, ¡, otros) 
\usepackage[utf8]{inputenc} 

% Incluimos la biblioteca para la inclusión de imágenes gif, png, jpeg. 
\usepackage{graphicx} 

% Asignamos un título, autor y fecha al documento
\title{Ejemplo de Documento} 
\author{Mi Nombre \\
             Departamento de Física \\
             Universidad de Sonora} 
% La doble diagonal invertida forza un cambio de renglón.
% Fija la fecha de hoy al documento  compilado. 
\date{\today} 
 
% Termina todo el preámbulo y sigue el inicio del documento con la instrucción \begin{document}
\begin{document}  
% La instrucción \begin{document} debe cerrarse al final del documento con \end{document} 

% Crea el título del documento, con autor y fecha incluidos. 
\maketitle 
    
% Crear una sección del documento, enumerada automáticamente.
\section{Primera Sección} 
\subsection{Primera Subsección}

% Testo del documento 
    
\textbf{Palabra en Negritas:} Este es un ejemplo de un programa fuente en \LaTeX.
% Notamos el . al final de la instrucción \LaTeX
     \LaTeX{} es un programa idel para elaborar \textit{textos (en itálicas)} que permite 
escribir ecuaciones
en un reglón tales como 
      $a^2+b^2=c^2$ 
% El símbolo $ le indica a  LaTeX que lo maneje como una expresión matemática.
También es posible escribir ecuaciones por separado:

% Crear un entorno matemático para escribir ecuaciones. 
    \begin{equation} 
    \gamma^2+\theta^2=\omega^2
    \end{equation}
Si no se insertan renglones en blanco \LaTeX{} continuará el texto hasta terminar el párrafo. 
 
Observen que aun cuando escribimos un párrafo en 
varios renglones, \LaTeX{} construye un párrafo de
longitud completa sin interrupción. Lo mismo        cualquiera     que sean   los    espacios   introducidos
en                        el                texto, estos      son ignorados.
 
Para comenzar un nuevo párrafo, dejamos una renglón en blanco. 
 
Cuando deseamos escribir listas, hay varias opciones: numeradas y no numeradas.
 
Ejemplo de lista sin enumerar:
\begin{itemize}
\item {Primer elemento}
\item {Segundo elemento}
\end{itemize}
 
Ejemplo de lista enumerada:
\begin{enumerate}
\item {Primer elemento}
\item {Segundo elemento}
\end{enumerate}
 
 
Para insertar imágenes, escribimos las siguientes instrucciones:

% Imagen centrada
\begin{center} 
\begin{figure}
    \centering
    \includegraphics[height=3cm]{Luna.jpg}
    \caption{Imágen de la Luna} 
\end{figure}   
\end{center}

% Este es el fin del documento (Debe corresponder al \begin{document} del principio).
\end{document} 
 
